\begin{foreignabstract}


The ATLAS experiment operates at the Large Hadron Collider, LHC, located in
CERN. Protons packages are injected into LHC and acellerated until its energy
increases by 15 times, summing up 7~TeV. The project goals require enormous data
acquisition rates. Therefore, the LHC has a peak crossing rate of 40~MHz. The
event rate can get up to 1~GHz after the upgrade that will start next year. This
way, it's crucial to filter online where only relevant information is observed
against the background noise within the processing speed requirements.

The muon identification is very important for exploring the entire physics
potential available with the ATLAS operation. It is possible to observe new
physics through the observation of such particles. The Muon Spectrometer is
installed on the ATLAS' surrond and it was designed to detect this elements.
However, some of the detector's region are not covered by its trigger stations.
In this areas, every muon is qualified as a low $p_T$ one, harming the detection
performance.

ATLAS will be upgraded in order to be prepared for an increase of the total
energy. Therefore, new technologies are proposed and evaluated. The current
research uses signal processing and computing intelligence techniques for aiding
the Level-1 trigger to recover these muons with high transverse momentum. To
this end, the calorimetry and the geometry information of the detector were
deemed.

\end{foreignabstract}

