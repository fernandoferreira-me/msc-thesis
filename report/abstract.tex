\begin{foreignabstract}

CERN is located in the border between Switzerland and France, near to Geneva
city. The LHC is operating there. It's is the world's largest and most powerful
particle acellerator. In the LHC, protons packages are injected and acellerated
until its energy increases by 15 times, summing up 7~TeV. The project goals
require enormous data acquisition rates. Therefore, the LHC has a peak crossing
rate of 40~MHz. The event rate can get up to 1~GHz after the upgrade that will
start next year. This way, it's crucial to filter online where only relevant
information is observed against the background noise within the processing speed
requirements.

The muon identification is very important for exploring the entire physics
potential available with the ATLAS operation. It is possible to observe new
physics through the observation of such particles. The Muon Spectrometer is
installed on the ATLAS' surrond and it was designed to detect this elements.
However, some of the detector's region are not covered by its trigger stations.
In this areas, every muon is qualified as a low $p_T$ one, harming the detection
performance.

This M.Sc. project use signal processing and computing intelligence techniques
for aiding the Level-1 trigger to recover these muons with high transverse momentum.

\end{foreignabstract}

