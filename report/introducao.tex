\chapter{Introdução}

A evolução e o desenvolvimento de técnicas de processamento de sinais e
inteligência computacional impulsionam diversas áreas do conhecimento. Pode-se
observar uma infinidade de aplicações nos mais diversos segmentos. Tais
ferramentas tendem a se tornar ainda mais poderosas quando utilizadas juntas.
Esta multidisciplinaridade característica, somada às inúmeras possibilidades de
abordagem de um problema, tornam a pesquisa nessas áreas instigante.

Este trabalho de pesquisa acontece no âmbito da detecção de sinais em ambientes
com baixa relação sinal-ruído. Experimentos dessa natureza apresentam um grande
volume de informações e elevadas dimensões em seu espaço amostral, dificultando
o processo de análise. A taxa de eventos tende a ser muito
elevada, pelo caráter raro dos sinais de interesse. Desta maneira, torna-se
fundamental a realização de uma filtragem \emph{online}, onde informações
relevantes são extraídas do ruído de fundo com velocidade condizente à taxa de
dados gerados.

Dependendo da aplicação, a aquisição de dados pode envolver detectores com
diferentes segmentações, contando com diversos sensores e canais, muitas vezes
distintos entre si.  Como consequência, são necessários sistemas robustos com
baixo custo computacional para contornar todas as condições hostis do projeto
especificado.

\section{Motivação}

A física experimental de altas energias é um dos ramos da ciência onde as
técnicas de processamento de sinais e inteligência computacional são mais
exigidas. Nela, procura-se comprovar experimentalmente teorias propostas pela
física teórica. Para tal, são construídos aparatos de proporções,
capazes de criar as condições ideais para a obtenção da física de interesse.

O LHC, instalado no CERN, é um acelerador capaz de atingir níveis de energia
nunca antes alcançados. O ATLAS, o maior dos detectores do acelerador, possui
milhões de canais de leitura, distribuídos por diferentes subsistemas,
utilizados para identificar partículas geradas com as colisões ocorridas durante
a operação.

Partindo do ponto de colisão em direção à área externa do ATLAS, encontram-se o
Detector de traços, utilizado para a identificação da trajetória de determinadas
partículas, o Calorímetro Eletromagnético e o Calorímetro Hadrônico, para a
análise da deposição energética de partículas eletromagnéticas e hadrônicas,
respectivamente, e o Espectrômetro de Múons, para a identificação
e determinação da trajetória de múons.

No ATLAS, ocorrem quarenta milhões de eventos por segundo, onde apenas duzentos
serão selecionados e armazenados para uma posterior análise \emph{offline}.
Deste modo, um sistema \emph{online} de filtragem eficiente é indispensável.
Esta seleção ocorre em 3 etapas através de níveis sequenciais:

\begin{enumerate}

    \item O primeiro nível do sistema de filtragem recebe todos os eventos
    gerados nas colisões. Neste ponto, apenas um em aproximadamente quinhentos e
    quarenta eventos é aceito.  A seleção é realizada através de \emph{hardware}
    de baixa granularidade utilizando informações de baixa resolução do sistema
    de calorimetria e do detector de múons, o que garante a velocidade de
    processamento necessária.

    Este nível também é responsável por gerar as chamadas Regiões de Interesse,
    áreas onde há informação relevante. Estas regiões são propagadas para o
    segundo nível;

    \item O segundo nível de filtragem considera apenas as regiões de interesse
    selecionadas na etapa anterior. Para tal, os algoritmos de filtragem têm
    acesso a resolução máxima do detector, bem como a trajetória percorrida
    pelas partículas.

    \item O terceiro nível de filtragem, conhecido como \emph{Event Filter},
    utiliza toda a informação adquirida para um evento. Ele é capaz de combinar
    as diferentes regiões de interesse para reconstruir os processos físicos
    ocorridos após a colisão.
\end{enumerate}

Com o advento do \emph{upgrade} do detector, previsto para a operação em 2020,
com etapas intermediárias em 2013 e 2017, novas arquiteturas podem ser propostas
e testadas~\cite{TSENG2008}.

\section{Objetivo}

Este trabalho de pesquisa utiliza técnicas de processamento de sinais e
inteligência computacional para auxiliar o primeiro nível de filtragem na
identificação de múons com alto valor de momento transverso.

Múons são partículas, eletricamente carregadas, duzentas vezes mais pesadas do
que os elétrons. Sua  identificação é crucial para explorar todo potencial
físico disponível através da operação do ATLAS. São de interesse do experimento
múons com poucos GeV até elevados patamares energéticos, onde é possível a
identificação de Nova Física~\cite{HASSANI2007}.

O Espectrômetro de Múons é instalado na camada externado ATLAS. Foi projetado
para detectar e reconstruir a trajetória de partículas que transpassam pelos
sistemas de calorimetria.

Por ser muito pesado, o múon dissipa pouca energia ao cruzar os detectores que
estejam ao longo de seu caminho. A estratégia do sistema de múons é aplicar um
campo magnético intenso, capaz de curvar a trajetória da partícula.  O múon
atravessará três camadas, compostas por câmaras resistivas,  ao longo de seu
percurso.  A posição em que a partícula cruza cada uma destas camadas determina
a curvatura de sua trajetória, que por sua vez é utilizada na determinação do
valor do momento linear. Quanto maior a energia cinética da partícula mais
suave será o seu raio de curvatura. Desta forma, entende-se que múons mais
energéticos obrigatoriamente atravessam as três camadas (ou estações de
coincidência) do espectrômetro, enquanto as menores podem passar apenas por
duas, devido ao desvio proporcionado pelo campo magnético aplicado. Este é,
portanto, um critério para distinção entre múons de alto momento e de baixo
momento.

Contudo, há regiões onde não há cobertura de câmaras nas três camadas. Nestas
áreas todos os múons são classificados automaticamente como pouco energéticos,
prejudicando a eficiência~\cite{ATLAS-CONF-2012-099}.

O Calorímetro de Telhas, por sua vez, é segmentado radialmente em três camadas.
Nelas, partículas hadrônicas, tais como prótons, são absorvidas. Contudo, em
virtude da espessura deste detector, a terceira camada possui baixa atividade de
interações com estas partículas, pois a maioria já foi absorvida nas duas
camadas mais internas. É nesta região que se localizam as chamadas células D.
Entende-se que as informações adquiridas nestas células  possam ser utilizadas
para mitigar o ruído de fundo na aquisição de múons.

Este trabalho utilizou as informações geométricas do calorímetro e do sistema de
múons, somadas a energia depositada pela passagem de partículas na célula D para
recuperar múons de momento elevado que não tiveram sua passagem registrada nas
três estações de coincidência.

\section{Organização do documento}

O próximo capítulo apresenta uma visão geral do ambiente do CERN, do experimento
ATLAS e seus sub-detectores.  O Capítulo~2 apresenta mais detalhadamente o
Calorímetro de Telhas e o Espectrômetro de Múons, pois a presente pesquisa foi
desenvolvida no âmbito dos dois sistemas.  O Capítulo~3 apresenta o sistema de
seleção de partículas do ATLAS. O nível 1 de filtragem, que utiliza informações
da calorimetria e do sistema de múons em seus algoritmos, é explicado em
detalhes. O Capítulo 4 mostra a base de dados utilizada para realizar as
análises e as diferentes abordagens adotadas. O Capítulo~5  é dedicado à
apresentação e discussão dos resultados obtidos.   Por fim, as conclusões, bem
como os possíveis desdobramentos deste trabalho de pesquisa, são apresentados no
Capítulo~6.
