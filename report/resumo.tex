\begin{abstract}

O CERN localiza-se na fronteira entre a Suíça e a França, próximo a cidade de
Genebra. Nele opera o LHC, o maior e mais poderoso acelerador de partículas do
mundo. No LHC, pacotes de prótons são injetados e acelerados até a energia
aumentar 15 vezes, adquirindo 7~TeV. Os objetivos do projeto demandam uma
elevada taxa de tomada de dados. Assim, o LHC opera a 40~MHz. Operando na
luminosidade máxima projetada para o colisor, a taxa de eventos poderá alcançar
1~GHz. Desta maneira, torna-se fundamental a realização de uma filtragem
\emph{online}, onde informações relevantes são extraídas do ruído de fundo com
velocidade condizente a taxa de dados gerados.

A identificação de múons de todos os patamares energéticos é crucial para
explorar todo potencial físico disponível através da operação do ATLAS. Através
da observação destas partículas é possível conhecer novas físicas. O
Espectrômetro de Múons é instalado na área externa do ATLAS e foi projetado para
detectar essas partículas.  Contudo, há regiões onde não há cobertura total de
estações. Nestas áreas todos os múons são classificados automaticamente como
pouco energéticos, prejudicando a eficiência de detecção.

Esta dissertação utiliza técnicas de processamento de sinais e
inteligência computacional para auxiliar o primeiro nível de filtragem a
recuperar múons com alto valor de momento transverso, erroneamente classificados.
\end{abstract}

