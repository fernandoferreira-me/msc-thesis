\chapter{Conclusões}

A identificação de múons de todos os patamares energéticos é crucial para
explorar todo potencial físico disponível através da operação do ATLAS. A partir
da observação destas partículas é possível conhecer novas físicas.

O Espectrômetro de Múons é instalado na camada externa do ATLAS e foi projetado
para detectar essas partículas. A estratégia deste sistema é aplicar um campo
magnético intenso, capaz de curvar a trajetória da partícula. Ao detectar a
passagem do alvo por suas três estações, o experimento consegue traçar o
caminho percorrido e, como consequência, estimar o momento transverso da
partícula. Contudo, há regiões onde não há cobertura de câmaras nas três
camadas. Nestas áreas todos os múons são classificados automaticamente como
pouco energéticos, prejudicando a eficiência de detecção.

Esta dissertação procurou recuperar os múons de momento elevado utilizando
informações geométricas do calorímetro e do sistema de múons, somadas à energia
depositada pela passagem de partículas na célula D. Esta célula localiza-se na
camada externa do TileCal e possui, portanto, baixa atividade hadrônica. Deste
modo, suas informações podem ser utilizadas para ajudar na detecção de múons.
Este trabalho é o desdobramento de~\cite{CIODARO2012}, que forneceu o estudo e
propôs um sistema capaz de melhorar a detecção de múons utilizando informações
de calorimetria. Este trabalho também apresentou uma metodologia para avaliar a
resposta do RPC à radiação de fundo da caverna do ATLAS.

Durante o desenvolvimento deste projeto, foram realizadas diversas reuniões com
a colaboração, o que possibilitou o melhor entendimento em relação ao sistema de
filtragem do L1, à seleção de múons e ao sistema de calorimetria. Também foram
disponibilizados dados de colisão ocorridos em 2011, o que viabilizou a
execução.

O Capítulo 4 apresentou inicialmente a metodologia utilizada durante o projeto.
Três abordagens foram realizadas:  estudo da deposição de energia de uma
partícula passante na célula D do TileCal; utilização da geometria do detector
ATLAS como fator classificatório; projeto de discriminadores neurais
utilizando os parâmetros disponíveis no L1, confirmados pela reconstrução
\emph{offline} de múons.

Através dos resultados pode-se observar que o depósito de energia na célula é
muito semelhante para todas as distribuições estudadas. A alta taxa de falso
alarme inviabiliza a utilização técnica de corte por patamar de energia sozinha.
Quando utilizou-se as informações de geometria do calorímetro logicamente
casadas à posição das regiões de interesse do RPC, obteve-se um pequeno ganho na
detecção, aliado à redução da taxa de falso alarme.

A utilização de classificadores neurais mitigou a taxa de falso alarme para
patamares condizentes com os requisitos de operação do L1. A taxa de detecção
obtida em ambos os conjuntos estudados indica que a célula D pode vir a ser uma
possível substituta da camada ausente do RPC. A Tabela~\ref{summary}
resume todos os resultados apresentados no Capítulo~5.

\begin{table}[hptp!]\footnotesize
  \centering
  \tabcolsep=0.08cm
  \begin{tabular}{ r c c c c c c }
       \multicolumn{1}{c}{} & \multicolumn{3}{c}{Detecção MU10}& \multicolumn{3}{c}{Detecção Múons Verdadeiros} \\
        \cmidrule(r){2-4}
        \cmidrule(r){5-7}
      & Dep. Energia & Geometria & Rede Neural & Dep. Energia & Geometria & Rede Neural \\
      \midrule
      Prob. Detecção & 60,3\% & 70,3\% & 75,6\% & 61,9\%  & 62,2\% & 82,3\% \\[1.5ex]
      Falso Alarme   & 50,7\% & 53,6\% & 15,4\% & 62,1\%  & 57,8\% & 21,2\% \\[1.5ex]
      Indice SP      & 0,55   & 0,58   & 0,80 & 0,49    & 0,51   & 0.82   \\
      \bottomrule
  \end{tabular}
  \caption{Tabela resumindo os resultados obtidos neste trabalho.}
  \label{summary}
\end{table}



À luz do \emph{upgrade} do detector, previsto para a operação em 2020, com
etapas intermediárias em 2013 e 2017, novas arquiteturas podem ser propostas. O
projeto e implantação de uma placa eletrônica que implemente as redes neurais
propostas nesse estudo no L1 é um dos desdobramentos possíveis. Antes, porém,
seria interessante um estudo mais detalhado da resposta do sistema proposto
frente à radiação de fundo do ATLAS.
